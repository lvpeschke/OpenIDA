%--------------------------------------------------------------------------
%	PACKAGES AND OTHER DOCUMENT CONFIGURATIONS
%--------------------------------------------------------------------------
\documentclass[11pt,a4paper]{article}
\usepackage[utf8]{inputenc}
\usepackage[english]{babel}
\usepackage[T1]{fontenc}
\usepackage{amsmath}
\usepackage{mathtools}
\usepackage{amsfonts}
\usepackage{amssymb}
\usepackage{pifont}% http://ctan.org/pkg/pifont
\usepackage{graphicx}
\usepackage{epstopdf}
\usepackage{lmodern}
\usepackage[left=3cm,right=3cm,top=2.5cm,bottom=2.5cm]{geometry}

\usepackage{fancyhdr} % Required for custom headers
\usepackage{lastpage} % Required to determine the last page for the footer
\usepackage{extramarks} % Required for headers and footers
\usepackage[usenames,dvipsnames]{color} % Required for custom colors
\usepackage{graphicx} % Required to insert images
\usepackage{caption}
\usepackage{subcaption}
\usepackage{listings} % Required for insertion of code
%\usepackage{courier} % Required for the courier font
\usepackage{verbatim}
\usepackage{multirow}
\usepackage{eurosym}
\usepackage{url}
\usepackage{hyperref}
\usepackage{color}
\usepackage[outline]{contour}
 \contourlength{.5pt}
\usepackage[noadjust]{cite}
\usepackage{tabularx}

\usepackage{enumerate}
%\usepackage{todonotes}
%\usepackage{relsize}

\usepackage{tikz}

\setlength\parindent{0pt} % Removes all indentation from paragraphs
\setlength{\parskip}{10pt plus 1pt minus 1pt}

%\definecolor{bleu}{HTML}{0000FF}
%\definecolor{jaune}{HTML}{EDB601}
%\definecolor{vert}{HTML}{008E45}
%\definecolor{rouge}{HTML}{FF0000}

%\renewcommand\thesection{\Roman{section}}

\begin{document}
	
%--------------------------------------------------------------------------
%	TITLE PAGE
%--------------------------------------------------------------------------
\begin{center}
{\bfseries
Linköping University\\
TDDD17 Information Security, Second Course\\

Lab 1: Authentication with OpenID\\
Lab assistant: Ulf Kargén\\[10pt]}

Guillaume Lambert (guila302) and Lena Peschke (lenpe782)\\
version 1, 15-01-27
\end{center}

\hrulefill

%--------------------------------------------------------------------------
%	CONTENT
%--------------------------------------------------------------------------

\section*{Our own authentication method}
\subsection*{Design of an authentication method}
\paragraph{Description}
% design choices
% -> combination of matrix positions, colours, (letters?)
% how it works

The authentication is based on a choice of matrix positions, colours, letters and password made at the registration.
This choice decides the answer of the challenges that are sent by the server to the user.

More concretely, the challenge consists of a randomly generated matrix 3x3 in which the 9 positions may contains a
different letter of color. The answer the user has to provide is the letters that fit his choices by either being in the
right position in the matrix, of the right color, or simply a chosen letter.

If no letters fit at least one of the chosen characteristics then the password has to be entered.

The parameters we chose are:
\begin{itemize}
\item 3 to 7 matrix positions
\item 2 or more colours among 6
\item 2 letters among the 26 capital letters
\end{itemize}

\paragraph{Risk analysis}
% security conclusions
% compare to passwords
% weaknesses + mitigate them

\paragraph{Use-case diagram}

\paragraph{Sequence diagram}

\subsection*{Implementation of an authentication method}
\paragraph{Overview}
% how to implement

\paragraph{Database}
\paragraph{User interface}
\paragraph{HTML and CSS}
\paragraph{Servlets}
\paragraph{OpenIDA}

\end{document}
